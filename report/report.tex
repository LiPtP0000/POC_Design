%!TEX program = xelatex
% 完整编译: xelatex -> biber/bibtex -> xelatex -> xelatex
\documentclass[lang=cn,a4paper,newtx]{elegantpaper}
\usepackage{algorithm}
\usepackage{algorithmicx}
\usepackage{algpseudocode}
\usepackage{subfig}
% \usepackage{subfigure}
% \usepackage{subfloat}
\title{计算机组织与结构II:POC实验报告}
\author{李勃璘 \\ 东南大学}

% \version{0.1}
\date{\zhdate{2025/3/18}}

% 本文档命令
\usepackage{array}
\newcommand{\ccr}[1]{\makecell{{\color{#1}\rule{1cm}{1cm}}}}
\addbibresource[location=local]{reference.bib} % 参考文献,不要删除

\begin{document}

\maketitle

\begin{abstract}
本文为计算机组织与结构II课程设计中,8位并行输出控制单元(Parallel Output Controller, POC) 的说明文档。该控制单元的状态和输入数据受一8位处理器控制,并可以与打印机交互,将内部数据打印出来。项目源代码已附在文末。
\end{abstract}

\section{整体功能}
本设计实现了一个8-bit的并行输出控制单元,其通过一个简易的8-bit处理器获取数据和控制信息,并将输出结果传输给打印机模块。打印机经过一段时间的延时后将数据打印出来。项目的总体结构如图~\ref{fig:structure}~所示。
\begin{figure}[htbp]
    \centering
    \includegraphics[width = 0.5\textwidth]{image/structure.png}
    \caption{项目模块结构}
    \label{fig:structure}
\end{figure}

\section{POC模块}
该模块完成并行输出控制单元(下文称POC)的描述,以及和打印机(下文称PRINTER)、处理器(下文称PROCESSOR)的交互。
\subsection{时钟与复位信号}
该模块的时钟和复位信号如表~\ref{tab:clk_poc}。
\begin{table}[htbp]
    \centering
    \renewcommand{\arraystretch}{1.2} % 调整行高
    \begin{tabular}{c c c c}
        \toprule
        信号名 & 位宽 & 流向 & 描述 \\
        \midrule
        i\_clk   & 1 & 输入 & 时钟信号\\
        i\_rst\_n & 1 & 输入 & 全局异步复位\\
        \bottomrule
    \end{tabular}
    \caption{POC模块的时钟与复位信号}
    \label{tab:clk_poc}
\end{table}
\subsection{模块接口}
该模块的对外接口如表~\ref{tab:port_description_poc}。
\begin{table}[htbp]
    \centering
    \renewcommand{\arraystretch}{1.2} % 调整行高
    \begin{tabular}{c c c c c}
        \toprule
        端口 & 位宽 & 流向 & 描述 & 来源/通向模块 \\
        \midrule
        i\_addr  &1 & 输入 & 地址信号,当值为0时,\texttt{i\_din}被解释为状态信号;& PROCESSOR \\
                & & & 当值为1时,被解释为要打印的数据 & \\
        o\_dout  & 8 & 输出 & 输出自身状态信号 & PROCESSOR \\
        i\_din   & 8 & 输入 & 数据信号,值意义参考\texttt{i\_addr}的值 & PROCESSOR\\
        i\_rw    & 1 & 输入 & 读写信号,当值为1时PROCESSOR向POC写值 & PROCESSOR \\
        i\_mode  & 1 & 输入 & 模式选择,1为中断模式,0为查询模式 & / \\
        o\_irq   & 1 & 输出 & 中断信号,为1时向PROCESSOR发出中断请求(低有效) & PROCESSOR \\
        i\_rdy   & 1 & 输入 & 信号为1时允许向打印机写数据 & PRINTER\\
        o\_tr    & 1 & 输出 & 信号为1时向打印机发送请求 & PRINTER\\
        o\_pd    & 8 & 输出 & 向打印机输出数据 & PRINTER \\
        \bottomrule
    \end{tabular}
    \caption{POC模块端口说明}
    \label{tab:port_description_poc}
\end{table}
\subsection{模块功能描述}
该模块描述了查询模式和中断模式下的POC状态转换和信号输入输出。模块功能见算法~\ref{alg:POC}。\footnote{表述和代码不完全相同。}实际实现时,为调试方便以及区分两种模式下的功能,为中断模式下的数据更新和打印机交互各设立了一个状态,但功能完全相同。


\begin{algorithm}[H]
    
    \caption{POC模块}
    \label{alg:POC}
    \begin{algorithmic}[1]
    \State \textbf{Note:} System turns query mode when $SR_0$ is 0, otherwise it turns to interrupt mode.
    \State \textbf{Input:} $i\_clk, i\_rst\_n, i\_mode, i\_rw, i\_addr, i\_din, i\_rdy$
    \State \textbf{Output:} $o\_dout, o\_irq, o\_tr, o\_pd$
    \State \textbf{Internal Registers:} $state, buffer, status, printer\_data, enable\_printer, ready$
    
    \Procedure{Initialize}{}
        \State $state \gets \text{Wait for PROCESSOR data}$
        \State $ready \gets 0$
        \State $status \gets 8'b10000000$
        \State $enable\_printer \gets 0$
        \State $interrupt \gets SR_7 \,\&\, SR_0$
        \State $SR_0 \gets i\_mode$
    \EndProcedure
    
    \Procedure{Wait for PROCESSOR data}{}
    \State \textbf{State Transition}:
        \If {$i\_rw == 1 \,\&\&\, SR_0 == 0\, or \, interrupt$}
            \State $state \gets \text{Update buffer and status from PROCESSOR}$
        \EndIf
    \EndProcedure
    
    \Procedure{Update buffer and status from PROCESSOR}{}
        \If {$i\_rw == 1$}
            \If {$i\_addr == 1$} 
                \State $buffer \gets i\_din$
            \ElsIf {$i\_addr == 0$} 
                \State $status \gets i\_din$
            \EndIf
        \EndIf
        \State \textbf{State Transition:}
        \If {$SR_7 == 0$} 
            \State $state \gets \text{Push buffer to PRINTER}$
        \EndIf
    \EndProcedure
    
    \Procedure{Push buffer to PRINTER}{}
        \If {$\text{PRINTER Ready}$}
            \State $printer\_data \gets buffer$ (Until the end of the state)
            \State $o\_tr \gets \text{pulse}$
            \State \textbf{State Transition:}
            \State $state \gets \text{Wait for PROCESSOR data}$
        \EndIf
    \EndProcedure
    \end{algorithmic}
    \end{algorithm}

\section{PROCESSOR模块}
\subsection{时钟与复位信号}
该模块的时钟和复位信号如表~\ref{tab:clk_processor}。
\begin{table}[htbp]
    \centering
    \renewcommand{\arraystretch}{1.2} % 调整行高
    \begin{tabular}{c c c c}
        \toprule
        信号名 & 位宽 & 流向 & 描述 \\
        \midrule
        i\_clk   & 1 & 输入 & 时钟信号\\
        i\_rst\_n & 1 & 输入 & 全局异步复位\\
        \bottomrule
    \end{tabular}
    \caption{POC模块的时钟与复位信号}
    \label{tab:clk_processor}
\end{table}
\subsection{模块接口}
该模块的对外接口如表~\ref{tab:port_description_cpu}。
\begin{table}[htbp]
    \centering
    \renewcommand{\arraystretch}{1.2} % 调整行高
    \begin{tabular}{c c c c c}
        \toprule
        端口 & 位宽 & 流向 & 描述 & 来源/通向模块 \\
        \midrule
        o\_addr  &1 & 输出 & 地址信号,当值为0时,\texttt{i\_din}被解释为状态信号;& POC \\
                & & & 当值为1时,被解释为要打印的数据 & \\
        i\_dout  & 8 & 输入 & POC的8位状态信号(SR) & POC \\
        o\_din   & 8 & 输出 & 数据信号,值意义参考\texttt{o\_addr}的值 & POC\\
        o\_rw    & 1 & 输出 & 读写信号,当值为1时PROCESSOR向POC写值 & POC \\
        i\_irq   & 1 & 输入 & 来自POC的中断信号(低有效) & POC \\
        i\_data  & 8 & 输入 & 系统输入数据(将被打印) & / \\
        \bottomrule
    \end{tabular}
    \caption{PROCESSOR模块端口说明}
    \label{tab:port_description_cpu}
\end{table}
\subsection{模块功能描述}
该模块描述了一个中央处理器(模块功能如算法~\ref{alg:PROCESSOR}~)。它向POC传递数据,并在传递数据结束后将$SR_7$归零以表征传送结束。处理器使用地址线选择要写入的寄存器,使用数据线写入数据。地址译码与数据分配由POC完成。

由于状态和数据共用一条传输线,有必要规定传送时序。规定当POC空闲时($SR_7$为1),先写入数据到传输线端口,然后打开写模式以更新传输线;更新之后等待一个周期,更改地址为“状态”,并写入更新$SR_7$后的状态数据。
\begin{algorithm}[H]
    \caption{PROCESSOR 模块}
    \label{alg:PROCESSOR}
    \begin{algorithmic}[1]
    \State \textbf{Note:} We denote STATUS as 0 and BUFFER as 1, as they represent the actual bit allocation.
    \State \textbf{Input:} $i\_clk, i\_rst\_n, i\_irq, i\_dout, i\_data$
    \State \textbf{Output:} $o\_din, o\_addr, o\_rw$
    \State \textbf{Internal Registers:} $poc\_status, address, rw, state, next\_state, set\_data\_done, read\_status\_done$
    
    \Procedure{Initialize}{ }

        \State $state \gets IDLE$
        \State $poc\_status \gets 8'b0$
        \State $read\_status\_done \gets 0$
    \EndProcedure
    \Procedure{State Transition}{ }
        \If{$state == IDLE$}
            \State $ (Mode == Interrupt)? state \gets READ\_FROM\_POC : state \gets SET\_DATA$
        \EndIf
        \State $ state \gets READ\_FROM\_POC \gets SET\_DATA$
        \State $ state \gets WRITE\_DATA \gets DELAY \gets WRITE\_STATUS $
        \State $ state \gets IDLE$
    \EndProcedure

    \Procedure{Control Logic for Address and Read-Write}{ }
        \State \textbf{state == IDLE: } $address \gets STATUS, rw \gets 0$

        \State \textbf{state == READ\_FROM\_POC: } $rw \gets 0, address \gets STATUS$

        \State \textbf{state == SET\_DATA: }$rw \gets 0, address \gets BUFFER$

        \State \textbf{state == WRITE\_DATA: } $rw \gets 1, address \gets BUFFER$

        \State \textbf{state == WRITE\_STATUS: }$rw \gets 1, address \gets STATUS$

    \EndProcedure
    
    \Procedure{Data Output Control}{ }
        \If {$address == STATUS$}
            \State $o\_din \gets poc\_status$
        \Else
            \State $o\_din \gets i\_data$
        \EndIf
    \EndProcedure
    
    \Procedure{Status Operations}{ }
        \If {$state == READ\_FROM\_POC$}
            \State $poc\_status \gets i\_dout$
        \ElsIf {$state == SET\_DATA$}
            \State $poc\_status[7] \gets 0$
        \EndIf

    \EndProcedure
    \end{algorithmic}
\end{algorithm}


\section{PRINTER模块}
\subsection{时钟与复位信号}
该模块的时钟和复位信号如表~\ref{tab:clk_poc}。
\begin{table}[htbp]
    \centering
    \renewcommand{\arraystretch}{1.2} % 调整行高
    \begin{tabular}{c c c c}
        \toprule
        信号名 & 位宽 & 流向 & 描述 \\
        \midrule
        i\_clk   & 1 & 输入 & 时钟信号\\
        i\_rst\_n & 1 & 输入 & 全局异步复位\\
        \bottomrule
    \end{tabular}
    \caption{PRINTER模块的时钟与复位信号}
    \label{tab:clk_printer}
\end{table}
\subsection{模块接口}
该模块的对外接口如表~\ref{tab:port_description_printer}。
\begin{table}[htbp]
    \centering
    \renewcommand{\arraystretch}{1.2} % 调整行高
    \begin{tabular}{c c c c c}
        \toprule
        端口 & 位宽 & 流向 & 描述 & 来源/通向模块 \\
        \midrule
        i\_tr  & 1 & 输入 & 冲激信号,象征打印开始 & POC \\ 
        i\_pd  & 8 & 输入 & POC的8位待打印数据,打印机将取冲激信号为1时刻的输入 & POC \\
        o\_rdy   & 1 & 输出 & 打印机工作时为0,否则为1 & POC\\
        o\_data    & 8 & 输出 & 打印机输出打印数据 & / \\
        \bottomrule
    \end{tabular}
    \caption{PRINTER模块端口说明}
    \label{tab:port_description_printer}
\end{table}
\subsection{模块功能描述}
该模块是一个打印机模块。在接收到来自POC的冲激信号之后,打印机将自身状态设置为BUSY。打印机需要8个时钟周期进行打印,打印的结果会延迟4个时钟。
\begin{algorithm}[H]
    \caption{PRINTER 模块}
    \label{alg:PRINTER}
    \begin{algorithmic}[1]
    \State \textbf{Note:} We denote STATUS as 0 and BUFFER as 1, as they represent the actual bit allocation.
    \State \textbf{Input:} $i\_clk, i\_rst\_n, i\_tr, i\_pd$
    \State \textbf{Output:} $o\_data, o\_rdy$
    

    \Procedure{IDLE}{ }
        \State Do nothing.
        \State $o\_rdy \gets 1$
        \State \textbf{State Transition: }
        \If {$i\_tr$ == 1}
            \State $o\_rdy \gets 0$ (with one clock cycle delay)
            \State $ state \gets BUSY$
        \EndIf
    \EndProcedure
    
    \Procedure{BUSY}{ }
        \If{ $counter == 8$}
            \State End of printing.
            \State $ state \gets IDLE$
        \Else
            \State Print data with a delay of 4 clock cycles.
            \State counter++;
        \EndIf
    \EndProcedure
    \end{algorithmic}
\end{algorithm}

\section{仿真验证}
\subsection{时延分析}
定义一个打印周期为从POC重置自身$SR_7$到一轮打印结束所经历的时钟上升沿个数。那么:
\begin{enumerate}
    \item 由POC重置到PROCESSOR置零$SR_7$,使用6个时钟,分别为读取POC信号的一个时钟和状态机的五个时钟;\footnote{中断模式较查询模式少用一个时钟。因为中断信号由组合逻辑生成,中断模式下PROCESSOR可立刻读取中断信号,无需读取POC信号。}
    \item POC此时开始与打印机进行交互。写入\texttt{o\_tr}和\texttt{o\_pd}共使用1个时钟,\texttt{o\_tr}信号持续1个时钟;
    \item 打印机收到信号后,耗时8个时钟打印数据;
    \item 打印结束后,POC重置$SR_7$,使用1个时钟。
\end{enumerate}
故一个打印周期为17个时钟(查询模式)或16个时钟(中断模式)。
\subsection{激励设置}
为了制造连续数据,避免数据丢包,输入数据的持续时间(延迟)应当和打印周期相同。仿真设置时钟以20ns为周期,故查询模式下,每一个数据更新之间的延迟应为340ns;中孤单模式下,每一个数据更新之间的延迟应为320ns。为简单起见,对两个不同的模式测试时,使用相同的测试样例。最终,选取无符号十进制数0-20按照340ns/320ns延时循环出现,作为仿真的输入数据。
\subsection{仿真结果}
依据激励设置,撰写POC模块的Testbench如附录~\ref{sec:tb}。使用iverilog+vvp+gtkwave运行仿真,运行结果为所有模块完整的波形图。

观察TOP模块的波形,可以看到查询和中断模式下(图~\ref{sim:top}~),信号都能够流水输出,这证明激励设置正确。中断模式下中断信号在每个周期开始时由POC置零一次,标志一个打印周期的开始;而查询模式下中断信号始终为高电平,这符合我们的预期。
\begin{figure}[htbp]
    \centering
    \subfloat[查询模式波形图]{\includegraphics[width = 0.95\textwidth]{image/overview_query.png}}
    \newline
    \centering
    \subfloat[中断模式波形图]{\includegraphics[width = 0.95\textwidth]{image/overview_interrupt.png}}
    \caption{TOP模块仿真波形图}
    \label{sim:top}
\end{figure}
% \nocite{*}
% \printbibliography[heading=bibintoc, title=\ebibname]

接下来观察各个子模块的波形,验证时延分析的分析结果。\texttt{i\_dout}是POC目前的状态,当POC将状态更新为1时,查询模式下PROCESSOR使用了6个时钟周期将$SR_7$更新为0。波形结果如图~\ref{fig:sim:processor}~所示。

\begin{figure}[htbp]
    \centering
    \includegraphics[width=0.95\textwidth]{image/query_pt1.png}
    \caption{PROCESSOR模块一个打印周期内的波形(查询模式)}
    \label{fig:sim:processor}
\end{figure}
\newpage
中断模式下,PROCESSOR使用了5个时钟周期将$SR_7$更新为0(图~\ref{fig:sim:processor_int}~)。
\begin{figure}[htbp]
    \centering
    \includegraphics[width=0.95\textwidth]{image/interrupt_pt1.png}
    \caption{PROCESSOR模块一个打印周期内的波形(中断模式)}
    \label{fig:sim:processor_int}
\end{figure}

POC的$SR_7$被更新为0后,POC使用一个时钟周期将\texttt{o\_tr}值赋为1。\texttt{o\_tr}持续了一个时钟周期(图~\ref{fig:sim:poc}~)。
\begin{figure}[htbp]
    \centering
    \includegraphics[width = 0.95\textwidth]{image/query_pt2.png}
    \caption{POC模块与PRINTER模块的交互}
    \label{fig:sim:poc}
\end{figure}

\texttt{o\_tr}的脉冲使得打印机开始打印。打印机选取了\texttt{o\_tr}为1时的待打印数据,延迟四个周期后输出到\texttt{o\_data}中。打印机在开始打印时即开始计数。计数器\texttt{counter}满8之后归零,打印机的\texttt{ready}信号重新置1。(图~\ref{fig:sim:printer}~)

\begin{figure}[htbp]
    \centering
    \includegraphics[width = 0.95\textwidth]{image/query_pt3.png}
    \caption{PRINTER模块打印捕获的数据}
    \label{fig:sim:printer}
\end{figure}

POC模块发现ready信号和寄存器中ready信号不同,即可判定打印结束。再经过一个周期后,POC模块将自身$SR_7$置1,打印周期结束。(图~\ref{fig:sim:poc2}~)
\begin{figure}[htbp]
    \centering
    \includegraphics[width = 0.95\textwidth]{image/query_pt4.png}
    \caption{POC模块更新$SR_7$}
    \label{fig:sim:poc2}
\end{figure}

综上所述,总时钟周期和理论分析相同,这代表硬件实现的结果和预期相符。
\newpage
\appendix
%\appendixpage
\section{完整设计代码}
\subsection{top.v}
\begin{lstlisting}[language=verilog]
module TOP (
    i_clk,
    i_rst_n,
    i_data,
    i_mode,
    o_tr,
    o_pd,
    o_rdy,
    o_data,
    o_rw,
    o_addr,
    o_irq,
    o_data_poc_to_processor,
    o_data_processor_to_poc
);
// I/O

input wire i_clk;
input wire i_rst_n;
input wire i_mode;
input wire [7:0] i_data;

output wire [7:0] o_data;
output wire o_irq;
output wire o_addr;
output wire o_rw;
output wire o_tr;
output wire [7:0] o_pd;
output wire o_rdy;
output wire [7:0] o_data_poc_to_processor;
output wire [7:0] o_data_processor_to_poc;
// CPU/POC connections

wire irq;
wire addr;
wire [7:0] data_processor_to_POC;
wire [7:0] data_POC_to_processor;
wire rw;

// POC/Printer connections

wire tr;
wire [7:0] pd;
wire rdy;

// Instantiations

PROCESSOR u_processor (
    .i_clk(i_clk),               // 时钟输入
    .i_rst_n(i_rst_n),           // 复位信号,低有效
    .i_irq(irq),                            // 中断
    .i_data(i_data),                        // Directly from TB
    .i_dout(data_POC_to_processor),         // 从 POC 读取的数据
    .o_din(data_processor_to_POC),          // 发送到 POC 的数据
    .o_addr(addr),                          // 地址信号
    .o_rw(rw)                  // 读/写信号
);

POC u_poc (
    .i_addr(addr),
    .o_dout(data_POC_to_processor),
    .i_din(data_processor_to_POC),
    .i_rw(rw),
    .i_clk(i_clk),
    .i_rst_n(i_rst_n),
    .i_mode(i_mode),
    .o_irq(irq),
    .i_rdy(rdy),
    .o_tr(tr),
    .o_pd(pd)
);

PRINTER u_printer (
    .i_tr(tr),
    .i_pd(pd),
    .o_rdy(rdy),
    .i_clk(i_clk),
    .i_rst_n(i_rst_n),
    .o_data(o_data)         // directly output to tb
);

// assignments

// CPU/POC 
assign o_irq = irq;
assign o_addr = addr;
assign o_rw = rw;
assign o_data_processor_to_poc = data_processor_to_POC;
assign o_data_poc_to_processor = data_POC_to_processor;

// POC/Printer 
assign o_tr = tr;
assign o_pd = pd;
assign o_rdy = rdy;


endmodule

\end{lstlisting}
\subsection{poc.v}
\begin{lstlisting}[language=verilog]
module POC (
    i_addr,
    o_dout,
    i_din,
    i_rw,
    i_clk,
    i_rst_n,
    i_mode,
    o_irq,
    i_rdy,
    o_tr,
    o_pd
);

  // From Top module

  input wire i_mode;  // 0 = query, 1 = interrupt

  // From/to Processor

  input wire i_addr;  // 0: status, 1: buffer    
  input wire i_clk;
  input wire i_rst_n;
  input wire i_rw;  // 0: read, 1: write
  input wire [7:0] i_din;  // status/data from Processor
  output wire [7:0] o_dout;  // status to CPU
  output wire o_irq;  // 0: interrupt

  // From/to Printer
  input wire i_rdy;
  output wire o_tr;
  output [7:0] o_pd;

  // Inside POC
  reg mode;
  reg [7:0] status;
  reg [7:0] buffer;
  reg [7:0] printer_data;
  reg enable_printer;
  wire interrupt;
  reg ready;



  // State Params

  parameter IDLE = 3'b0;
  // Wait for CPU response
  parameter POLLING_CPU_WRITE = 3'b001;
  // POC will receive data from processor using polling method
  parameter POLLING_TO_PRINTER = 3'b010;
  // POC will transmit data to printer
  parameter INTERRUPT_CPU_WRITE = 3'b011;
  // POC will receive data from processor using interrupt method
  parameter INTERRUPT_TO_PRINTER = 3'b100;
  // POC will transmit data to printer


  reg [2:0] state, next_state;


  // initial 
  always @(posedge i_clk or negedge i_rst_n) begin
    if (!i_rst_n) begin
      state <= IDLE;
      ready <= 'b0;  // maybe there is a more beautiful solution
      mode  <= 'b0;
    end else begin
      state <= next_state;
      ready <= i_rdy;
      mode  <= i_mode;
    end
  end

  // state transition
  always @(*) begin
    case (state)
      IDLE: begin
        if (i_rw == 1'b1) begin
          if (status[0] == 0) begin  // query
            next_state = POLLING_CPU_WRITE;
          end else begin
            if (interrupt == 1'b1) begin
              next_state = INTERRUPT_CPU_WRITE;
            end
          end
        end else begin
          next_state = IDLE;
        end
      end
      POLLING_CPU_WRITE: begin
        if (status[7] == 0) begin
          next_state = POLLING_TO_PRINTER;
        end else begin
          next_state = POLLING_CPU_WRITE;
        end
      end
      POLLING_TO_PRINTER: begin
        if (i_rdy == 'b1 && ready == 'b0) begin
          next_state = IDLE;
        end else begin
          next_state = POLLING_TO_PRINTER;
        end
      end
      INTERRUPT_CPU_WRITE: begin
        if (status[7] == 0) begin
          next_state = INTERRUPT_TO_PRINTER;
        end else begin
          next_state = INTERRUPT_CPU_WRITE;
        end
      end
      INTERRUPT_TO_PRINTER: begin
        if (i_rdy == 'b1 && ready == 'b0) begin
          next_state = IDLE;
        end else begin
          next_state = INTERRUPT_TO_PRINTER;
        end
      end
      default: begin
        next_state = IDLE;
      end
    endcase
  end


  // status and buffer
  always @(*) begin
    case (state)
      IDLE: begin
        status         = {7'b1000000, mode};  // reset to POC ready
        printer_data   = 8'b0;  // every cycle reset the printer data
        enable_printer = 1'b0;  // every cycle reset the printer status
      end
      POLLING_CPU_WRITE: begin
        // CPU needs to write both buffer and status in this state
        if (i_rw == 1'b1 && i_addr == 1'b1) begin
          buffer = i_din;
        end
        if (i_rw == 1'b1 && i_addr == 1'b0) begin
          status = i_din;
        end
      end
      INTERRUPT_CPU_WRITE: begin
        if (i_rw == 1'b1 && i_addr == 1'b1) begin
          buffer = i_din;
        end
        if (i_rw == 1'b1 && i_addr == 1'b0) begin
          status = i_din;
        end
      end
    endcase
  end

  always @(posedge i_clk or negedge i_rst_n) begin
    if (!i_rst_n) enable_printer <= 1'b0;
    else begin
      if (state == POLLING_TO_PRINTER || state == INTERRUPT_TO_PRINTER) begin

        if (ready == 1'b1 && enable_printer == 1'b0) begin
          printer_data   <= buffer;
          enable_printer <= 1'b1;
        end else if (enable_printer == 1'b1) begin
          enable_printer <= 1'b0;
        end
      end
    end
  end

  // Assignments
  assign o_irq = ~interrupt;
  assign o_tr = enable_printer;
  assign o_pd = printer_data;
  assign o_dout = status;

  assign interrupt = status[7] & status[0];

endmodule

\end{lstlisting}
\subsection{processor.v}
\begin{lstlisting}[language=verilog]
module PROCESSOR (
    i_clk,
    i_rst_n,
    i_irq,
    i_dout,
    i_data,
    o_din,
    o_addr,
    o_rw
);


  input wire i_clk;
  input wire i_rst_n;
  input wire i_irq;
  input wire [7:0] i_dout;  // Data from POC (specifically status info)
  input wire [7:0] i_data;  // Data from top module

  output reg [7:0] o_din;  // Data to POC
  output wire o_addr;
  output wire o_rw;

  // Internal registers
  reg [7:0] poc_status;
  reg address;
  reg rw;

  // State registers
  reg [2:0] state, next_state;

  // Status tracking signals
  reg set_data_done;
  reg read_status_done;

  // Mode signal from poc_status[0]
  wire mode = poc_status[0];
  wire interrupt = i_irq;

  // State encoding
  parameter STATUS = 1'b0;
  parameter BUFFER = 1'b1;

  parameter IDLE = 3'b000;
  parameter READ_FROM_POC = 3'b001;
  parameter SET_DATA = 3'b010;
  parameter WRITE_DATA = 3'b011;
  parameter DELAY = 3'b100;
  parameter WRITE_STATUS = 3'b101;

  // State transition
  always @(posedge i_clk or negedge i_rst_n) begin
    if (!i_rst_n) state <= IDLE;
    else state <= next_state;
  end

  // Next state logic
  always @(*) begin
    case (state)
      IDLE: begin
        if (mode == 1'b0) next_state = READ_FROM_POC;
        else if (interrupt == 1'b0) next_state = SET_DATA;
        else next_state = IDLE;
      end
      READ_FROM_POC: begin
        if (read_status_done && poc_status[7] == 1'b1) next_state = SET_DATA;
        else next_state = READ_FROM_POC;
      end
      SET_DATA: begin
        if (set_data_done) next_state = WRITE_DATA;
        else next_state = SET_DATA;
      end
      WRITE_DATA: begin
        next_state = DELAY;  // wait one clock cycle for poc read delay
      end
      DELAY: begin
        next_state = WRITE_STATUS;
      end
      WRITE_STATUS: begin
        next_state = IDLE;
      end
      default: next_state = IDLE;
    endcase
  end

  // Control logic for address and rw
  always @(*) begin
    case (state)
      IDLE: begin
        address = STATUS;
        rw = 1'b0;
      end
      READ_FROM_POC: begin
        rw = 1'b0;
        address = STATUS;
      end
      SET_DATA: begin
        rw = 1'b0;
        address = BUFFER;
      end
      WRITE_DATA: begin
        rw = 1'b1;
        address = BUFFER;
      end
      WRITE_STATUS: begin
        rw = 1'b1;
        address = STATUS;
      end
    endcase
  end

  // Output data control
  always @(*) begin
    set_data_done = 1'b0;
    if (state == WRITE_DATA || state == IDLE) begin
      set_data_done = 1'b0;
    end else begin
      if (address == STATUS) o_din = poc_status;
      else begin
        o_din = i_data;
        set_data_done = 1'b1;
      end
    end
  end

  // poc_status and read_status_done update
  always @(posedge i_clk or negedge i_rst_n) begin
    if (!i_rst_n) begin
      read_status_done <= 1'b0;
      poc_status <= 8'b0;
    end else begin
      case (state)
        IDLE: begin
          read_status_done <= 1'b0;
        end
        READ_FROM_POC: begin
          poc_status <= i_dout;
          read_status_done <= 1'b1;
        end
        SET_DATA: begin
          poc_status[7] <= 1'b0;
        end
      endcase
    end
  end

  // Assign output signals
  assign o_rw   = rw;
  assign o_addr = address;

endmodule

\end{lstlisting}
\subsection{printer.v}
\begin{lstlisting}[language=verilog] 
module PRINTER (
    i_tr,
    i_pd,
    o_rdy,
    i_clk,
    i_rst_n,
    o_data
);
  // interfaces
  input wire i_clk;
  input wire i_rst_n;
  input wire i_tr;
  input wire [7:0] i_pd;
  output wire o_rdy;
  output wire [7:0] o_data;

  // inside printer
  reg [7:0] delay_buffer[0:3];  // for delaying 4 cycles
  reg [2:0] count;
  reg state, next_state;

  parameter IDLE = 1'b0;
  parameter BUSY = 1'b1;


  always @(posedge i_clk or negedge i_rst_n) begin
    if (!i_rst_n) begin
      state <= IDLE;
    end else begin
      state <= next_state;
    end
  end

  // state transition
  always @(*) begin
    case (state)
      IDLE: begin
        if (i_tr) begin
          next_state <= BUSY;
        end else begin
          next_state <= IDLE;
        end
      end
      BUSY: begin
        // we will wait 8 clock cycles for printer to print, even though the printing process only needs 4.
        if (count == 3'b111) begin
          next_state <= IDLE;
        end else begin
          next_state <= BUSY;
        end
      end
      default: begin
        next_state <= IDLE;
      end
    endcase

  end

  // counter
  always @(posedge i_clk or negedge i_rst_n) begin
    if (!i_rst_n) begin
      count <= 3'b0;
    end else begin
      case (state)
        IDLE: begin
          count <= 3'b0;
        end
        BUSY: begin
          count <= count + 1;
        end
        default: begin
          count <= 3'b0;
        end
      endcase
    end
  end

  // delay four cycles to simulate printing process
  always @(posedge i_clk) begin
    if (state == BUSY) begin
      delay_buffer[0] <= i_pd;
    end
  end

  always @(posedge i_clk) begin
    if (state == BUSY) begin
      delay_buffer[1] <= delay_buffer[0];
    end
  end

  always @(posedge i_clk) begin
    if (state == BUSY) begin
      delay_buffer[2] <= delay_buffer[1];
    end
  end

  always @(posedge i_clk) begin
    if (state == BUSY) begin
      delay_buffer[3] <= delay_buffer[2];
    end
  end

  assign o_data = delay_buffer[3];
  assign o_rdy  = (state == IDLE) ? 1'b1 : 1'b0;

endmodule
    
\end{lstlisting}
\subsection{tb.v}\label{sec:tb}
\begin{lstlisting}[language=verilog]
    // Date: 2025.3.10
    // Author: LiPtP
    // Description: Testbench for TOP module
    `timescale 1ns / 1ps
    module tb_POC;
    
      // Clock and reset
      reg clk;
      reg reset;
    
      // Input to modules
      reg [7:0] data_in;
      reg mode;
    
      // Output from modules
      wire [7:0] printer_data;
      wire printer_tr;
      wire [7:0] poc_pd;
      wire poc_ready;
    
      wire rw;
      wire addr;
      wire [7:0] data_POC_to_processor;
      wire [7:0] data_processor_to_POC;
      wire irq_n;
    
      // Clock generation (50MHz, 20ns period)
      always #10 clk = ~clk;
    
      // Reset sequence
      initial begin
        clk = 0;
        reset = 0;
        mode = 0;
        data_in = 8'b0;
    
        #50 reset = 1;  // 50ns delay to allow reset propagation
      end
    
      // Mode switching after 100,000 ns
      initial begin
        #100000 mode = 1;
      end
    
      // Periodic Data Input
      initial begin
        #100;
        while (1) begin
          if (mode == 0) begin
            data_in = 8'b00000000;
            #340;
            data_in = 8'b00000001;
            #340;
            data_in = 8'b00000010;
            #340;
            data_in = 8'b00000011;
            #340;
            data_in = 8'b00000100;
            #340;
            data_in = 8'b00000101;
            #340;
            data_in = 8'b00000110;
            #340;
            data_in = 8'b00000111;
            #340;
            data_in = 8'b00001000;
            #340;
            data_in = 8'b00001001;
            #340;
            data_in = 8'b00001010;
            #340;
            data_in = 8'b00001011;
            #340;
            data_in = 8'b00001100;
            #340;
            data_in = 8'b00001101;
            #340;
            data_in = 8'b00001110;
            #340;
            data_in = 8'b00001111;
            #340;
            data_in = 8'b00010000;
            #340;
            data_in = 8'b00010001;
            #340;
            data_in = 8'b00010010;
            #340;
            data_in = 8'b00010011;
            #340;
            data_in = 8'b00010100;
            #340;
          end else begin
            data_in = 8'b00000000;
            #320;
            data_in = 8'b00000001;
            #320;
            data_in = 8'b00000010;
            #320;
            data_in = 8'b00000011;
            #320;
            data_in = 8'b00000100;
            #320;
            data_in = 8'b00000101;
            #320;
            data_in = 8'b00000110;
            #320;
            data_in = 8'b00000111;
            #320;
            data_in = 8'b00001000;
            #320;
            data_in = 8'b00001001;
            #320;
            data_in = 8'b00001010;
            #320;
            data_in = 8'b00001011;
            #320;
            data_in = 8'b00001100;
            #320;
            data_in = 8'b00001101;
            #320;
            data_in = 8'b00001110;
            #320;
            data_in = 8'b00001111;
            #320;
            data_in = 8'b00010000;
            #320;
            data_in = 8'b00010001;
            #320;
            data_in = 8'b00010010;
            #320;
            data_in = 8'b00010011;
            #320;
            data_in = 8'b00010100;
            #320;
    
          end
        end
      end
    
      // Dump waveforms
      initial begin
        $dumpfile("poc.vcd");
        $dumpvars(0, tb_POC);
        #200000 $finish;
      end
    
      // Instantiate DUT
      TOP u_dut (
          .i_clk(clk),
          .i_rst_n(reset),
          .i_data(data_in),
          .i_mode(mode),
          .o_tr(printer_tr),
          .o_pd(poc_pd),
          .o_rdy(poc_ready),
          .o_data(printer_data),
          .o_rw(rw),
          .o_addr(addr),
          .o_irq(irq_n),
          .o_data_poc_to_processor(data_POC_to_processor),
          .o_data_processor_to_poc(data_processor_to_POC)
      );
    
    endmodule
    
\end{lstlisting}
\addappheadtotoc

\end{document}
